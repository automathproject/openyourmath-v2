\uuid{CXE4}
\chapitre{Autre}
\sousChapitre{Autre}
\titre{ Rétropropagation à une variable }
\theme{réseaux de neurones}
\auteur{Maxime NGUYEN}
\datecreate{2024-12-02}
\organisation{AMSCC}

\contenu{

\texte{ 
	 Vous allez travailler avec un réseau de neurones à une couche cachée, conçu pour effectuer une tâche de prédiction simple. 
	
	
	
	\textbf{Description du réseau}
	
	\begin{itemize}
		\item \textbf{Entrée :}
		\begin{itemize}
			\item \( x \in \mathbb{R} \) (scalaire)
		\end{itemize}
		
		\item \textbf{Couche cachée (2 neurones) :}
		\begin{itemize}
			\item \textbf{Neurone caché 1 :}
			\begin{itemize}
				\item Poids : \( w_1 = 0{,}15 \)
				\item Biais : \( b_1 = 0{,}35 \)
			\end{itemize}
			\item \textbf{Neurone caché 2 :}
			\begin{itemize}
				\item Poids : \( w_2 = 0{,}20 \)
				\item Biais : \( b_2 = 0{,}35 \)
			\end{itemize}
		\end{itemize}
		
		\item \textbf{Couche de sortie (1 neurone) :}
		\begin{itemize}
			\item \textbf{Neurone de sortie :}
			\begin{itemize}
				\item Poids : \( w_3 = 0{,}40 \) (connecté au neurone caché 1)
				\item Poids : \( w_4 = 0{,}45 \) (connecté au neurone caché 2)
				\item Biais : \( b_3 = 0{,}60 \)
			\end{itemize}
		\end{itemize}
		
		\item \textbf{Fonctions d'activation :}
		\begin{itemize}
			\item Fonction sigmoïde : \( \sigma(z) = \dfrac{1}{1 + e^{-z}} \)
		\end{itemize}
		
		\item \textbf{Fonction de coût :}
		\begin{itemize}
			\item Erreur quadratique moyenne : \( E = \dfrac{1}{2}(y - \hat{y})^2 \)
		\end{itemize}
		
		\item \textbf{Données :}
		\begin{itemize}
			\item \textbf{Entrée :} \( x = 0{,}05 \)
			\item \textbf{Sortie cible :} \( y = 0{,}01 \)
		\end{itemize}
	\end{itemize}
 }

 \begin{enumerate}
	\item \question{Calculer les sorties des neurones de la couche cachée (\( h_1 \) et \( h_2 \)).}
	\indication{}
    \reponse{
    	Neurone caché 1 : 
    	\begin{align*}
    	\text{Entrée nette : } & net_{h1} = w_1 \cdot x + b_1 \\
    	& net_{h1} = 0{,}15 \times 0{,}05 + 0{,}35 = 0{,}3575 \\
    	\text{Sortie : } & h_1 = \sigma(net_{h1}) = \dfrac{1}{1 + e^{-0{,}3575}} \approx 0{,}5889
    	\end{align*}
    	
    	
    	Neurone caché 2 :
    	
    	
    	\begin{align*}
    	\text{Entrée nette : } & net_{h2} = w_2 \cdot x + b_2 \\
    	& net_{h2} = 0{,}20 \times 0{,}05 + 0{,}35 = 0{,}36 \\
    	\text{Sortie : } & h_2 = \sigma(net_{h2}) = \dfrac{1}{1 + e^{-0{,}36}} \approx 0{,}5890
    	\end{align*}
    	}
    \item \question{Calculer la sortie du neurone de sortie (\( \hat{y} \)).}
    \indication{}
    \reponse{ 
    	\begin{align*}
    	\text{Entrée nette : } & net_o = w_3 \cdot h_1 + w_4 \cdot h_2 + b_3 \\
    	& net_o = 0{,}40 \times 0{,}5889 + 0{,}45 \times 0{,}5890 + 0{,}60 \\
    	& net_o = 0{,}2356 + 0{,}2650 + 0{,}60 = 1{,}1006 \\
    	\text{Sortie : } & \hat{y} = \sigma(net_o) = \dfrac{1}{1 + e^{-1{,}1006}} \approx 0{,}7507
    	\end{align*}
    	 }
    \item \question{ Calculer l'erreur du réseau en utilisant la fonction de coût. }
    \reponse{ 
    	\begin{align*}
    	E = \dfrac{1}{2}(y - \hat{y})^2 = \dfrac{1}{2}(0{,}01 - 0{,}7507)^2 = \dfrac{1}{2}(-0{,}7407)^2 \approx 0{,}2741
    	\end{align*}
    	
     }
    \item \question{ Calculer les gradients des poids \( w_3 \) et \( w_4 \) de la couche de sortie  les gradients des poids \( w_1 \) et \( w_2 \) de la couche cachée. }
    \reponse{ Calcul de l'erreur locale au neurone de sortie (\( \delta_o \)) :
    	
    	
    	\begin{align*}
    	\delta_o = \dfrac{\partial E}{\partial net_o} = \dfrac{\partial E}{\partial \hat{y}} \cdot \dfrac{\partial \hat{y}}{\partial net_o} \\
    	\dfrac{\partial E}{\partial \hat{y}} = \hat{y} - y = 0{,}7507 - 0{,}01 = 0{,}7407 \\
    	\dfrac{\partial \hat{y}}{\partial net_o} = \hat{y}(1 - \hat{y}) = 0{,}7507 \times 0{,}2493 \approx 0{,}1875 \\
    	\delta_o = 0{,}7407 \times 0{,}1875 \approx 0{,}1389
    	\end{align*}
    	
    	
    	Calcul des gradients des poids :
    	
    	
    	\begin{align*}
    	\dfrac{\partial E}{\partial w_3} = \delta_o \cdot h_1 = 0{,}1389 \times 0{,}5889 \approx 0{,}0817 \\
    	\dfrac{\partial E}{\partial w_4} = \delta_o \cdot h_2 = 0{,}1389 \times 0{,}5890 \approx 0{,}0818
    	\end{align*}
    	
    	
    	Calcul des gradients des poids \( w_1 \) et \( w_2 \)
    	
    	Calcul des erreurs locales aux neurones cachés (\( \delta_{h1} \) et \( \delta_{h2} \)) :
    	
    	
    	\begin{align*}
    	\delta_{h1} = \delta_o \cdot w_3 \cdot h_1 (1 - h_1) \\
    	& = 0{,}1389 \times 0{,}40 \times 0{,}5889 \times (1 - 0{,}5889) \\
    	& \approx 0{,}1389 \times 0{,}40 \times 0{,}5889 \times 0{,}4111 \approx 0{,}0134 \\
    	\delta_{h2} = \delta_o \cdot w_4 \cdot h_2 (1 - h_2) \\
    	& = 0{,}1389 \times 0{,}45 \times 0{,}5890 \times (1 - 0{,}5890) \\
    	& \approx 0{,}1389 \times 0{,}45 \times 0{,}5890 \times 0{,}4110 \approx 0{,}0150
    	\end{align*}
    	
    	
    	Calcul des gradients des poids :
    	
    	
    	\begin{align*}
    	\dfrac{\partial E}{\partial w_1} = \delta_{h1} \cdot x = 0{,}0134 \times 0{,}05 \approx 0{,}0007 \\
    	\dfrac{\partial E}{\partial w_2} = \delta_{h2} \cdot x = 0{,}0150 \times 0{,}05 \approx 0{,}0008
    	\end{align*}
    	 }
    \item \question{ Avec un taux d'apprentissage \( \eta = 0{,}5 \), mettre à jour tous les poids du réseau. }
    \reponse{ Avec \( \eta = 0{,}5 \) :
    	
    	- Mise à jour des poids de la couche de sortie :
    	
    	
    	\begin{align*}
    	w_3^{\text{nouveau}} = w_3^{\text{ancien}} - \eta \dfrac{\partial E}{\partial w_3} = 0{,}40 - 0{,}5 \times 0{,}0817 = 0{,}3592 \\
    	w_4^{\text{nouveau}} = w_4^{\text{ancien}} - \eta \dfrac{\partial E}{\partial w_4} = 0{,}45 - 0{,}5 \times 0{,}0818 = 0{,}4091
    	\end{align*}
    	
    	
    	- Mise à jour des poids de la couche cachée :
    	
    	
    	\begin{align*}
    	w_1^{\text{nouveau}} = w_1^{\text{ancien}} - \eta \dfrac{\partial E}{\partial w_1} = 0{,}15 - 0{,}5 \times 0{,}0007 = 0{,}1497 \\
    	w_2^{\text{nouveau}} = w_2^{\text{ancien}} - \eta \dfrac{\partial E}{\partial w_2} = 0{,}20 - 0{,}5 \times 0{,}0008 = 0{,}1996
    	\end{align*}
    	 }

\end{enumerate}

}

