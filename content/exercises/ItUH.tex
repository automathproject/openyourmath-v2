\uuid{ItUH}
\chapitre{Probabilité continue}
\sousChapitre{Loi normale}
\titre{Propriétés de la loi normale centrée réduite}
\theme{loi normale}
\auteur{Maxime NGUYEN}
\datecreate{2022-10-17}
\organisation{AMSCC}
\contenu{

\texte{ Soit  $Z$ une variable aléatoire suivant une loi normale centrée réduite $\mathcal{N}(0,1)$. } 

\question{ Justifier que $\PP(Z>0) = \frac{1}{2}$ et que pour tout réel $a \in \R$, $\PP(Z<-a) = \PP(Z>a)$. }

\reponse{ On sait que $\PP(Z>0) = \int_0^{+\infty} \frac{1}{\sqrt{2\pi}}\;\; \mathrm{e}^{-\frac{x^2}{2}} dx = \frac{1}{2} \int_{-\infty}^{+\infty} \frac{1}{\sqrt{2\pi}}\;\; \mathrm{e}^{-\frac{x^2}{2}} dx$ par parité de la fonction intégrée. Or $\int_{-\infty}^{+\infty} \frac{1}{\sqrt{2\pi}}\;\; \mathrm{e}^{-\frac{x^2}{2}} dx$ d'où le premier résultat.

De même, on a par changement de variable $u=-x$ et parité : $$\PP(Z<-a) = \int_{-\infty}^{-a} \frac{1}{\sqrt{2\pi}}\;\; \mathrm{e}^{-\frac{x^2}{2}} dx = -\int_{+\infty}^{a} \frac{1}{\sqrt{2\pi}}\;\; \mathrm{e}^{-\frac{u^2}{2}} du = \int^{+\infty}_{a} \frac{1}{\sqrt{2\pi}}\;\; \mathrm{e}^{-\frac{u^2}{2}} du = \PP(Z>a)$$
  }
}
