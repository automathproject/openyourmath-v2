\uuid{kJz6}
\chapitre{Probabilité discrète}
\sousChapitre{Probabilité et dénombrement}
\titre{Probabilité d'événements}
\theme{probabilités}
\auteur{ }
\datecreate{2023-08-30}
\organisation{AMSCC}
%Proba360

\contenu{
\texte{ Soit $(\Omega, \mathcal{T}, \prob)$ un espace probabilisé et deux événements $A$ et $B$. On note $\overline{A}$ et $\overline{B}$ les événements complémentaires de $A$ et $B$. 
	
	On sait que : $$\prob(A) = \frac{1}{2} \quad , \quad \prob(A \cup B) = \frac{7}{8} \quad , \quad \prob(\overline{B}) = \frac{3}{8}$$
	


Calculer les probabilités suivantes : 
\begin{enumerate}
	\item \question{ $\prob(A \cap B)$ ; }
	\reponse{  $\prob(A \cap B) = \frac{2}{8}$ }
	\item \question{ $\prob(\overline{A} \cap \overline{B})$ ; }
	\reponse{  $\prob(\overline{A} \cap \overline{B}) = \frac{1}{8}$ }
	\item \question{ $\prob(\overline{A} \cup \overline{B})$ ; }
	\reponse{ $\prob(\overline{A} \cup \overline{B}) = \frac{6}{8}$ }
	\item \question{ $\prob(\overline{A} \cup B)$. }
	\reponse{ $\prob(\overline{A} \cup B) = \frac{3}{8}$ }
\end{enumerate}
}
}
