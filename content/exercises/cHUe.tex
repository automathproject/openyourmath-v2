\uuid{cHUe}
\chapitre{Statistique}
\sousChapitre{Tests d'hypothèses, intervalle de confiance}
\titre{Test d'indépendance }
\theme{statistiques, tests d'hypothèses}
\auteur{}
\datecreate{2023-12-06}
\organisation{AMSCC}

\contenu{
\texte{ On  interroge 1873 ́étudiants  de  Master  sur  la  catégorie  socio-professionnelle  de  leurs parents.  Les  ́étudiants  suivent  différents  cursus :   ́écoles  d'ingénieurs,  ́école de  commerce,  universités scientifiques,  médecine.  Les résultats sont les suivants :
	
\begin{center}
		\begin{tabular}{|c|c|c|c|c|}
		\hline 
		& Ouvriers & Employés & Cadres & Professions libérales \\ 
		\hline 
		Ecoles d'ingénieurs & 50 & 280 & 120 & 20 \\ 
		\hline 
		Ecoles de commerce & 8 & 29 & 210 & 350 \\ 
		\hline 
		Universités scientifiques & 150 & 230 & 100 & 40 \\ 
		\hline 
		Médecine & 26 & 80 & 80 & 100 \\ 
		\hline 
	\end{tabular} 
\end{center}

On s'intéresse à l'influence du milieu socio-professionnel des parents sur le type d'étude des enfants. Pour répondre à cette question : }

\begin{enumerate}
	\item \question{ Réaliser un test d'indépendance du $\chi^2$ en expliquant le résultat obtenu pour un risque de première espèce de $5\%$. }
	
	\item \question{ Jusqu'à quel risque de première espèce peut-on maintenir cette décision ? Le résultat obtenu s'appelle la $p$-valeur du test. }
\end{enumerate}
}