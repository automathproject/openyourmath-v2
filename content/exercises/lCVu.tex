\uuid{lCVu}
\chapitre{Déterminant, système linéaire}
\sousChapitre{Calcul de déterminants}
\titre{Déterminant d'un produit}
\theme{calcul déterminant}
\auteur{}
\datecreate{2023-01-11}
\organisation{AMSCC}
\contenu{

\texte{ Soient $A$ et $B$ deux matrices carrées telles que $\det(AB)=-2, \det(B)>0$, et $B=B^{-1}$. }

\question{Déterminer $\det(A)$. }


\reponse{ On a $\operatorname{det}(A \cdot B)=\operatorname{det}(A) \times \operatorname{det}(B)=-2$, et $B=B^{-1} \Rightarrow \operatorname{det}(B)=\operatorname{det}\left(B^{-1}\right)$.
Or $\operatorname{det}\left(B^{-1}\right)=(\operatorname{det}(B))^{-1}$, aussi :
$$
\operatorname{det}(B)=\operatorname{det}\left(B^{-1}\right)=(\operatorname{det}(B))^{-1}=\frac{1}{\operatorname{det}(B)} \Leftrightarrow(\operatorname{det}(B))^2=1 \Leftrightarrow \operatorname{det}(B)=\pm 1
$$
Puisque $\operatorname{det}(B)>0, \operatorname{det}(B)=+1$ et $\operatorname{det}(A)=-2$. }}
