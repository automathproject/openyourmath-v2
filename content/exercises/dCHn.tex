\uuid{dCHn}
\chapitre{Probabilité discrète}
\sousChapitre{Probabilité conditionnelle}
\titre{Fonctionnement d'une machine}
\theme{probabilités conditionnelles}
\auteur{}
\datecreate{2023-09-04}
\organisation{AMSCC}
\contenu{

\texte{ Une usine a besoin de deux machines $m_1$ et $m_2$. La probabilité que $m_1$ tombe en panne est $0{,}005$ et la probabilité que $m_2$ tombe en panne est $0{,}007$. La probabilité que $m_2$ tombe en panne sachant que $m_1$ est en panne est $0{,}5$. }

\reponse{Soit $M_1$ (respectivement $M_2$) l'événement ``la machine $m_1$ (respectivement $m_2$) fonctionne''. On a ainsi:
	$\prob(\overline{M}_1)=0{,}005$, $\prob(\overline{M}_2)=0{,}007$, $\prob(\overline{M_2}|\overline{M}_1)=0{,}5$.
}

\begin{enumerate}
	\item \begin{enumerate}
		\item \question{ Quelle est la probabilité que $m_1$ et $m_2$ soient simultanément en panne ? }
		 \reponse{ $\prob(\overline{M}_1\cap \overline{M}_2)=\prob(\overline{M}_1)\prob(\overline{M}_2|\overline{M}_1)=0{,}5\times 0{,}005=0{,}0025$.
		}
		
		\item \question{ En déduire la probabilité qu'une machine au moins fonctionne. }
		 \reponse{$\prob(M_1\cup M_2)=\prob(\overline{\overline{M}_1\cap \overline{M}_2})=1-\prob(\overline{M}_1\cap \overline{M}_2)=1-0{,}0025=0{,}9975$
		}
		
	\end{enumerate}
	\item 
	\begin{enumerate}
		\item \question{ Quelle est la probabilité que $m_1$ (respectivement $m_2$) soit seule en panne ? }
		 \reponse{Probabilité que $m_1$ soit seule en panne :
			$$ \prob(\overline{M}_1\cap M_2)=\prob(M_2| \overline{M}_1)\,  \prob(\overline{M}_1)=0{,}005\times 0{,}5=0{,}0025.$$
			On peut aussi voir que $\overline{M}_1\cap M_2 = \overline{M}_1 \setminus \overline{M}_1\cap \overline{M}_2$, ce qui donne $\prob(\overline{M}_1\cap M_2) = 0{,}005-0{,}0025 = 0{,}0025$. 

			Probabilité que $m_2$ soit seule en panne :
			\begin{eqnarray*}
				\prob(\overline{M}_2\cap M_1) = 1-\prob(\overline{M_1}\cup M_2) &=& 1-(\prob(\overline{M}_1)+\prob(M_2)-\prob(\overline{M}_1\cap M_2))\\
				&=& 1- (0{,}005+0{,}993-0{,}0025) = 0{,}0045
			\end{eqnarray*}
			On peut aussi voir comme ci-dessus que $\overline{M}_2\cap M_1 = \overline{M}_2 \setminus \overline{M}_2\cap \overline{M}_1$, ce qui donne également que $\prob(\overline{M}_2\cap M_1) = 0{,}007-0{,}0025 = 0{,}0045$. 
		}
		
		\item \question{ En déduire la probabilité d'avoir une seule machine en panne. }
		 \reponse{Soit $A$ l'événement : ``une seule machine est en panne''. Alors 
			\[ \prob(A)=\prob(M_1\cap \overline{M}_2)+\prob(M_2\cap \overline{M}_1)=0{,}025+0{,}0045=0{,}007\]
		}
		
		\item \question{ Quelle est la probabilité de n'avoir aucune machine en panne ? }
		 \reponse{ 
			L'événement qui nous intéresse est $M_1\cap M_2$. Or on a
			\begin{eqnarray*}
				\prob(M_1\cap M_2) = 1- \prob(\overline{M}_1\cup \overline{M}_2)
				&=& 1- (\prob(\overline{M}_1)+\prob(\overline{M}_2)-\prob(\overline{M}_1\cap \overline{M}_2))\\
				&=& 1-(0{,}005+0{,}007-0{,}0025) = 0{,}9905.
			\end{eqnarray*}
			Il y a donc $99{,}05$\% de chances que les deux machines fonctionnent.
			
			On peut aussi voir que $A$, $\overline{M}_1\cap \overline{M}_2$ et $M_1\cap M_2$ forme un système complet d'événements : soit 0 machine fonctionne, soit une seule fonctionne, soit les deux fonctionnent. Ainsi, on retrouve 
			$$\prob(M_1 \cap M_2)=1-\prob(A)-\prob(\overline{M}_1\cap\overline{M}_2)=1-0{,}007-0{,}0025=0{,}9905.$$
		}
		
	\end{enumerate}
\end{enumerate}
}